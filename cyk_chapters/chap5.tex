\chapter{总结和展望}
\section{总结}

本文以工业自动化零件装配为背景,针对其中物体的六维位姿估计展开研究。学习研究过程中主要设计了以下几个主要方面:集成学习算法、仿真深度相机、以及物体级联位姿估计等。本文针对集成学习中的随机蕨和随机森林算法展开了较为深入的研究,提出了对随机蕨算法的一些改进和补充。本文还基于OpenGL软件平台设计了一套深度相机仿真环境,并利用该仿真环境提供的大量仿真深度图像数据进行了物体的位姿估计算法流程开发,采用基于随机蕨的算法框架可以实现工业现场刚性零件的物体实时监测以及位姿估计。主要贡献点有如下几点:

\begin{itemize}
\item 归纳整理了随机蕨算法应用于回归问题的解决方案,并提出了一种面向大噪声干扰增加掩码机制的改进随机蕨回归算法。随机蕨算法在研究初期被用于解决分类问题,具有比随机森林算法更为优秀的精度以及效率,但缺点在于不能应用于回归问题。本文通过总结前人经验,给出了随机蕨算法应用于回归问题的解决方案。并针对实际问题中很多特征输入存在较大干扰的情况,提出了一套改进的随机蕨算法。经过本文改进后的随机蕨算法可以利用较易获取的特征置信度掩码进行有选择作出修正,使得随机蕨算法在遇到局部大噪声干扰的情况下仍然可以保持较好的回归精准度,提升了鲁棒性。最后通过一些实验验证了改进后算法的可行性以及准确性。

\item 搭建了一套深度相机仿真环境平台,通过借助OpenGL工具,能够实现在短时间内给出大量带有真值标注的仿真深度图像数据。这些仿真数据可以用于训练对应工业零件的位姿估计模型,从而为物体位姿估计算法开发提供了高效方便的基础环境。该仿真环境能够预先加载工业现场的深度背景信息,同时也可以模拟复杂环境下物体被随机遮挡的情况,使得最后得到的仿真深度图像更为真实以及丰富,从而让位姿估计模型的训练结果与真实数据训练出的模型更为接近。最后还提出了如何对真实深度相机采集到的深度图像中的空洞进行修补的办法,进一步缩小了仿真深度图像与真实深度图像之间差别。在实验部分本文给出了一些仿真深度图像,表明了该方针深度图像的合理性。

\item 提出了一种像素差特征的图像特征描述方法和采用改进随机蕨回归的物体六维位姿估计算法。首先提出了一种基于深度图像中的像素差特征的全新深度图像特征描述方法。该特征描述方法具有非常高的计算效率,同时具有尺度不变性,能够很好捕捉深度图像中指定位置的物体位姿信息。然后采用有监督学习的特征哈希表达方法,显著提高原始特征的信噪比,使得物体位姿估计模型的训练更加鲁棒。随后借助本文提出的随机蕨回归算法,提出了一套针对物体位姿估计的级联回归算法框架,修正了普通级联回归算法框架中回归目标过于固定的不足。最后还利用改进后的随机蕨回归算法,提出了一种遮挡情况下物体位姿估计算法,以及利用位姿回归所用的特征描述进行物体检测,形成一套完整的从物体定位到物体位姿估计算法框架。本文还通过大量实验测试了整套算法的可行性,以及给出了一些定量数据表明算法在速度和精度上都取得了一定的成绩。
\end{itemize}


\section{展望}

本文虽然提出了一个较为系统全面的物体位姿估计方案,并取得了一定的成果,但是在算法框架的很多实现细节上仍然有较大的改进空间,主要有如下几点:

\begin{itemize}
\item 目前改进后的随机蕨算法在训练的过程中依旧需要调试大量的参数,参数的不合理会导致模型过于庞大或者是模型的拟合回归能力不足。因此,后续工作可以针对自适应参数的随机蕨算法开发进行展开。

\item 在进行深度相机仿真的环节中,本文通过添加随机位置的前景信息模拟物体被遮挡的情况。此处由于没有加入物理引擎,会导致一些模拟出的前景障碍物实际与仿真物体没有接触,导致仿真效果不是非常理想。因此,如果能够加入物理引擎,在物理引擎工作之后再捕捉仿真深度图像是可以得到更好的仿真深度数据的。

\item 本文设计的用于物体位姿估计的特征描述有着标准的正方形形状,而实际应用场景中,被检测的物体形状往往不是规则的正方形形状。对于长条形或者其他不规则形状的物体,本文的特征描述效果就会大打折扣。因此如何设计一个能够适用于各种形状特征描述方法是今后研究工作的一个重要方向。

\item 利用位姿估计的特征进行物体检测时,本文验证性的采用了基本滑动窗口的方式进行。其中由于特征描述的范围较大,并且特征描述的范围大小随着深度值的改变而改变,因此滑动窗口的方向和大小的设定应该可以根据实际深度值进行。这样的改进应该可以大大减少滑动窗口的密度,从而提高算法的速度。
\end{itemize}






















