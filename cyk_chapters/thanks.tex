\begin{thanks}
时间飞逝,转眼就在浙江大学控制科学与工程学院机器人实验室度过了两年半时间里,我将永远不会忘记来自导师、同学、 朋友和家人的指导、帮助、关心和理解。你们的支持是我一直以来的安心科研的动力。
在此期间首先要衷心感谢我的导师,熊蓉教授。

从本科大一就开始着迷于机器人小制作的我,通过熊老师带头组织的“中控杯”机器人竞赛更深刻地体会到了机器人能给我带来的快乐。我也因此走上了研究机器人领域的求学之路,并在本科期间就得到了熊老师细心的指导和点拨。进入研究生后,熊老师依旧不遗余力地给我们传授经验和知识,同时给我以及整个实验室同学提供了绝对丰富的科研平台,让我能够顺利地进行各种实验。除此之外,熊老师总是能够在我科研遇到瓶颈的时候给我带来如母亲般地关怀,并且能给出非常好的建议和启发,带我从低谷中爬出。不光在科研上我能够在熊老师的指导下快速成长,每周一次的组会熊老师必定亲自指导我们如何展示自己的科研工作,如何清晰地表达自己的想法和成果。以及在文献撰写中,每一次都能得到熊老师细心的点评,让我从中学到了严谨的治学态度。生活中,熊老师也特别和蔼可亲,能够在团建活动中给我们带来食堂里吃不到的美味佳肴,还常常在出差途中不忘给我们捎带当地特色的礼物,分外感动。机器人实验室大家庭里,熊老师一直都是我们的姐姐,在实验室里的生活也将成为我一生中珍贵的记忆。

其次,也要感谢浙江大学机器人实验室里与我朝夕相处的每一位同学,你们的陪伴和帮助给了我前进 的动力。尤其感谢方立师兄、誉洪生师兄、王亚彪师兄、王鹏师兄、胡晋师兄、王军南师兄以及赵永生师兄在平时学术上给予的大力指导,从你们身上我学到的东西甚至超过了平时上课所学的内容。同时也要感谢王文、陈颖、李东轩、唐立、王浩、吴珺等和我一起进入实验室的挚友们,与你们相处的过程中我受益颇深。我将永远不会忘记与你们在一起的时光。

最后,感谢父母的养育之恩,也是你们一步一步的引导和教育才能让我慢慢走到今天。感谢周围所有理解我支持我的同学和益友,是你们的陪伴让我的生活充满阳光。

\end{thanks}
