\chapter{基于OpenGL的深度相机仿真环境搭建}

\section{概述}
大数据驱动下的模型等训练需要大量带有人工标注真值的数据作为基础。深度相机作为近几年流行起来的新型相机,能够获取除了颜色意外的额外的视觉信息,极大促进了物体位姿估计领域的研究发展。物体位姿估计领域内越累越多的基于深度相机采集的深度图像的算法被提出并取得了不错的效果。但是这些物体位姿估计算法很难应用到实际生产生活中去,有一个很大的原因在于这些算法都依赖带有真值标注等训练数据,而这些带有真值的训练数据非常难以获取。有真值标注的训练数据的可获取性和易获取程度极大的影响到了物体位姿估计算法的进一步发展,因此本章将针对这一点,讨论一个基于OpenGL平台的深度相机仿真环境。

本章的主旨在于能够通过任意指定一个刚性物体的3D数据文件,便可以借助OpenGL计算模拟深度相机,自动并且快速地给出大量带有真值标注的深度相机仿真深度数据。由于本文研究内容主要在于位姿估计方面,因此本文给出的深度相机仿真环境给出的数据主要有仿真深度图像以及给定物体在仿真相机下的6维位姿$P$。其中这6维位姿包括以相机参考系为参考系下的物体的欧几里得坐标($P_x,P_y,P_z$),以及相机参考系下的欧拉角位姿($P_ax,P_ay,P_az$),如下图所示:

图图图图图图





\section{仿真数学模型}
\subsection{投影变换}
\subsection{坐标系以及坐标系} % 四元数、参考系的转换等 
\subsection{OpenGL原理} % stl文件、深度检测等等

\section{仿真环境搭建}
\subsection{仿真流程} % 
\subsection{模拟深度数据噪声} % 双高斯

\section{实验结果与分析}
