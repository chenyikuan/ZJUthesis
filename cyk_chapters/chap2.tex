\chapter{随机蕨算法的改造与应用}

\section{概述}
随机蕨算法是随机森林算法的一种变种,也是基于传统贝叶斯理论提出的一种集成学习方法,
最早由Mustafa Ozuysal等人提出\cite{ozuysal2007fast}。
随机蕨和随机森林的区别主要有如下这四个方面:
\begin{itemize}
\item
随机森林算法直接学习后验概率$P(C_k|F)$,
而随机蕨算法则是通过学习类别条件概率密度分布$P(F|C_k)$
\item
在随机森林算法中,对于每个输入样本,比较的特征次序会有所不同,
但是在随机蕨算法中,对于每个输入的样本比较的特征的次序都是相同的。
\item
在训练过程中,随机森林算法的训练时间是随着随机树的深度指数增长,
而在随机蕨算法中,训练时间仅仅随着数的深度呈线性增长。
\item
在随机森林算法中,算法的最后结果是通过对每个决策树的加权平均来得到的,
但是在随机蕨算法中,最后结果则是以贝叶斯规则综合得到。
\end{itemize}
其中,$P$表述算法最后的概率输出结果,$C_k$表示算法分类器的目标类别,
$F=\{f_1,f_2,...,f_N\}$表示输入算法的特征,下文同。

首先先简要阐述一下随机蕨算法的具体原理。
对于一个分类问题,需要计算各个类别间的概率分布:
\begin{equation}
	\arg\max_{k} P(C_k|f_1,f_2,...,f_N)
\end{equation}
在贝叶斯规则下,上式等价于:
\begin{equation}
	\arg\max_{k} P(f_1,f_2,...,f_N|C_k)P(C_k)
\end{equation}
该式子表明一个后验概率与先验概率密度与相似函数的乘积成正比。
然而对于一个高纬度的特征描述,或者说一个复杂的决策树输入样本,
是很难计算出或者学习出相关联的相似度分布。
或者说计算高维度的相似度分布是一件非常耗时耗资源的工作,
在很多计算资源有限、计算实时要求高的应用场景中,计算高维特征相似度
变得非常笨重,$P(f_1,f_2,...,f_N|C_k)$会非常难以求解。但如果假设在给定了目标类别的情况下,
特征的各个维度之间是有一定的相互独立性的。这个时候,
朴素贝叶斯理论可以对上述公式有如下简化:
\begin{equation}
	P(f_1,f_2,...,f_N|C_k)=\prod_{i=1}^N P(f_i|C_k)
\end{equation}
从而目标类别表达式为:
\begin{equation}
	Class(F)\equiv \arg\max_k P(C_k)\prod_{n=1}^N P(f_n|C_k)
\end{equation}
但这个独立性的假设在大多数的情况下是很难成立的,大多数情况下,上式得到的概率结果
会小于真实的后验概率。为此,需要进一步假设。假设高维特征可以分解为几个小特征组合
,并且这些小的特征组合之间是几乎完全相互独立的。这个假设在很多情况下都可以满足实用性。
例如原始特征$F$分解为L组,每一组的特征大小为S,则有:
\begin{equation}
\begin{aligned}
	F=\{F_1,F_2,...,F_L\} \\
% \end{equation}
% \begin{equation}
	F_l=\{f_{l_1},f_{l_2},...,f_{l_S}\}
\end{aligned}
\end{equation}
当L组特征间相互独立假设下:
\begin{equation}
	P(f_1,f_2,...,f_N|C_k)=\prod_{l=1}^L P(F_l|C_k)
\end{equation}
于是可以得到最后的类别概率分布:
\begin{equation}
	Class(F)\equiv \arg\max_k P(C_k)\prod_{l=1}^L P(F_l|C_k)
\end{equation}
上式就是随机蕨算法的根本,采用半朴素贝叶斯的方式,平衡了算法计算的复杂性以及
算法的准确性。其中每一个特征组构成的决策树被称为一个蕨,实际应用过程中,通过
调整蕨的大小(也就是$S$的大小),就可以实现对算法复杂性以及准确性的控制。

上述问题的讨论都是在分类问题的基础上进行的,算法最后的结果也是在给定类别数目
的情况下才能实现。但是在很多实际问题中,往往遇到的是回归问题,需要求解的目标
空间往往是连续的,这个时候上述方法将不再适用,本文将针对这样的情况,结合随机
蕨的特性,借鉴回归算法在随机森林算法中的应用,将随机蕨算法加以改造,使其适用于拟合
回归问题。在此基础上,本文将进一步扩充随机蕨算法,通过增加一层掩码机制
使其在输入特征有较大噪声的情况下也能有较强的鲁棒性,
使算法能够在更多的应用场景中有更稳定的表现。

本章结构安排如下:2.2节介绍随机蕨算法通过修改输出函数表达式可以实现其在拟合回归
问题中的应用。2.3介将介绍如何将掩码添加进随机蕨算法框架中,实现算法在对抗大噪声
情况下仍然具有较强的鲁棒性。最后在2.4节,本文通过将随机蕨算法应用到人脸对齐问题中,
通过实验验证了随机蕨算法在回归问题中的可行性以及相比于普通拟合算法的鲁棒性,
说明了该算法具有较强的实用性。


\section{随机蕨在拟合回归问题中的应用改造}
目前随机蕨

\section{大噪声干扰下的随机蕨回归算法}

\section{实验结果与分析} % 人脸回归可以在这里写




























