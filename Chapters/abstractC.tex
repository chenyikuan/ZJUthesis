\begin{abstract}
随着工业机器人领域的不断发展,工业现场的演示编程技术成为了非常前沿的研究课题。演示编程提供了一种新的向机器人传递信息的方式,是简化机器人编程的 重要途径。与传统机器人编程方法相比,它可以大大降低行业应用技术人员在机器人使用和编程方面所需的专业性知识要求,对于机器人的推广应用具有重要意义。物体的六维位姿估计是演示编程技术中一个非常重要的环节,但目前没有一套较为实用的物体位姿估计算法,使得很多柔性制造工业环节效率低下,因此本文提出了一套简单易用的物体六维位姿估计算法框架。在对该算法框架的研究开发过程中,主要有如下几个工作内容:

\begin{itemize}
\item 通过研究集成学习方法,着重探讨了集成学习算法中随机蕨算法这个分支。随机蕨算法在研究初期被用于解决分类问题,具有比随机森林算法更为优秀的精度以及效率,但缺点在于不能应用于回归问题。本文通过总结前人经验,归纳整理了给出了随机蕨算法应用于回归问题的解决方案。此外,结合了实际问题中很多特征输入存在较大干扰的情况,提出了一套改进的随机蕨算法。经过本文改进后的随机蕨算法可以利用较易获取的特征置信度掩码进行有选择作出修正,使得随机蕨算法在遇到局部大噪声干扰的情况下仍然可以保持较好的回归精准度,提升了鲁棒性。

\item 为了能够使物体位姿估计算法有更高的应用价值,或者说为了能够在短时间内训练出一个新零件的位姿估计模型,本文通过借助OpenGL软件平台,搭建了一套深度相机仿真环境,用于快速给出大量的带有真值标注的仿真深度图像数据。该仿真环境能够预先加载工业现场的深度背景信息,同时也可以模拟复杂环境下物体被随机遮挡的情况,使得最后得到的仿真深度图像更为真实以及丰富,从而让位姿估计模型的训练结果与真实数据训练出的模型更为接近。最后还提出了如何对真实深度相机采集到的深度图像中的空洞进行修补的办法,进一步缩小了仿真深度图像与真实深度图像之间差别。

\item 研究了一套全新的物体六维位姿估计算法框架。首先提出了一种基于深度图像中的像素差特征的全新深度图像特征描述方法。该特征描述方法具有非常高的计算效率,同时具有尺度不变性,能够很好捕捉深度图像中指定位置的物体位姿信息。然后简要介绍了一下有监督学习的特征哈希表达方法。该哈希表达方法能够显著提高原始特征的信噪比,使得物体位姿估计模型的训练更加鲁棒。随后借助本文提出的随机蕨回归算法,提出了一套针对物体位姿估计的级联回归算法框架,修正了普通级联回归算法框架中回归目标过于固定的不足。最后还利用改进后的随机蕨回归算法,提出了一种遮挡情况下物体位姿估计算法,以及利用位姿回归所用的特征描述进行物体检测,形成一套完整的从物体定位到物体位姿估计算法框架。
\end{itemize}



\keywords{演示编程,位姿估计,随机蕨,仿真深度相机,物体检测}
\end{abstract}
