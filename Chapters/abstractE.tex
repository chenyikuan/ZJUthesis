\begin{englishabstract}

With the continuous development of industrial robots, the object six-dimensional pose estimation becomes an important research topic, which can improve the industrial automation. Estimating pose of industrial parts through an automated vision system can simplify the deployment of robots' point of operation and greatly liberate manpower so as to improve the efficiency of industrial production. Adressing the problem of parts random distribution and partial occlusion in manufacturing industry, a simple and easy-to-use algorithm of six-dimensional pose estimation algorithm is proposed in this paper. The main idea of the algorithm is to build a cascade regression framework through simulation environment. And the main contents and achievements of this paper are as follows:


% With the continuous development of industrial robots, simplifying robot programming, reducing dependency of professional, speeding up the programming speed have become important research topics in the field of industrial robots. Programming by demonstration provides a new way to communicate information to robots and is an important way to simplify the robot programming. Comparing with the traditional robot programming method, it can greatly reduce the professional knowledge required by industrial application technicians in robotic use and programming, which is of great significance for the popularizing of robots. Pose estimation is a very important part of programming by demonstration. Estimating the pose of industrial parts through an automated vision system can simplify the deployment of robotic operations points in robot programming and increases the efficiency of industrial production. However, there is no economical object pose estimation algorithm yet. So a simple and easy-to-use algorithm for estimating the six-dimensional pose of objects is proposed in this paper. The core idea of ​​this algorithm is of great help to the development of programming by demonstration. The main contents and achievements of this paper are as follows:


\begin{itemize}
\item By studying the integrated learning method, we focus on the branch of random fern algorithm. The random fern algorithm is used to solve the classification problem in the early stage of the study, and has more excellent precision and efficiency than the random forest algorithm. But the disadvantage is that it can not be applied to the regression problem. In this paper, by summarizing the experience of predecessors, we proposed a modified fern algorithm whitch can be applied to regression problem. In addition, a advanced random fern algorithm is proposed based on the fact that many features input have large interference and cause disaster to regression tasks. The advanced random fern algorithm can be adaptively modified by using the easily accessible feature confidence mask, so that the modified random fern algorithm can still maintain good regression accuracy and robustness.

\item In order to make the object pose estimation algorithm get higher application value, or to be able to train a pose estimation model of a new part in a short time, this paper use the OpenGL software platform to build a depth camera simulation environment, in order to efficiently produce a large number of labeled depth image data. The simulation environment can pre-load the background information of industrial scene and also can simulate the objects occlusion. Because of the resulting simulation depth image is more real and rich, the training results of pose estimation model is closer to the model trained by real data. In the end, this paper also proposes how to repair the holes in the depth image collected by the real depth camera in order to further reduce the difference between the simulated depth image and the true depth image.

\item A novel algorithm of six-dimensional pose estimation algorithm is proposed in this paper. First of all, a new method of depth image feature description based on pixel difference features in depth image is discussed. The feature description method is very computational efficient and alse has scale invariance feature. It can well capture the object location and orientation information of the specified position in the depth image. Then the supervised hashing method is briefly introduced. The hashing method can significantly improve the signal-to-noise ratio of the original feature and make the training of object pose estimation model more robust. Then, with the help of the random fern regression algorithm proposed in this paper, a  cascade regression algorithm framework for object pose estimation is proposed, which fixes the problem that the regression target in the general cascade regression algorithm framework is too restricted. Finally, an improved random fern regression algorithm is also proposed to handle object pose estimation under occlusion. And the object detection is performed by using the feature descriptors used in pose regression. Totally, all this methods forms a complete object pose estimation algorithm framework.
\end{itemize}

\englishkeywords{Pose estimation, Random ferns, Simulation of depth camera, Object detection}

\end{englishabstract}
